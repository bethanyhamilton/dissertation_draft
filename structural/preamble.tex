
%\usepackage[authoryear, round]{natbib}
\usepackage{styles/utformat}  		% Package style file for UT ETDs.


%%%%%%%%%%%%%%%%%%%%%%%%%%%%%%%%%%%%%%%%%%%%%%%%%%%%%%%%%%%%%%%%%%%%%%
% Optional packages used for this template. If you don't
% need a capability in your document, feel free to comment out/remove
% the package usage command.
%%%%%%%%%%%%%%%%%%%%%%%%%%%%%%%%%%%%%%%%%%%%%%%%%%%%%%%%%%%%%%%%%%%%%%
\usepackage{hyperref}

\usepackage{xurl}
\usepackage{academicons}
\usepackage{xcolor}
\newcommand{\orcid}[1]{\href{https://orcid.org/#1}{\includegraphics{orcid_logo}}}


\usepackage{pdflscape}
% \usepackage{layout}
\usepackage{graphicx}
\usepackage{amsmath,amsthm,amsfonts,amscd}
				% Some packages to write mathematics.
\usepackage{eucal} 	 	% Euler fonts
\usepackage{verbatim}      	% Allows quoting source with commands.
\usepackage{styles/citesort}         	%
\usepackage{algorithm}
\usepackage{bibentry}
\usepackage{algorithmic}
\usepackage{xcolor}
\usepackage{markdown}
\usepackage{booktabs}
\usepackage{url}

\usepackage{multirow}
\usepackage{makecell}
\usepackage{rotating}
\usepackage{siunitx}
\usepackage{tabularx}
\usepackage{bbm}% Allows good typesetting of web URLs.
\usepackage{mathtools}
\usepackage{hyperref}
\hypersetup{colorlinks=false}

\usepackage[english]{babel}
\usepackage{csquotes}
\usepackage[style=apa,sortcites=true]{biblatex}

\usepackage[fulladjust]{marginnote}
\setlength{\marginparsep}{5mm}
\setlength{\marginparwidth}{1in}

\usepackage{bm}
\usepackage{empheq}
%%%%%%%%%%%%%%%%%%%%%%%%%%%%%%
% Default is one-and-a-half spacing. Double is also permitted.%
%%%%%%%%%%%%%%%%%%%%%%%%%%%%%%
\oneandonehalfspacing
% \doublespacing

%%%%%%%%%%%%%%%%%%%%%%%%%%%%%%%%%%%%%%%%%%%%%%%%%%%%%%%%%%%%%%%%%%%%%%
%	Some math support.					     %
%%%%%%%%%%%%%%%%%%%%%%%%%%%%%%%%%%%%%%%%%%%%%%%%%%%%%%%%%%%%%%%%%%%%%%
%
%	Theorem environments (these need the amsthm package)
%
%% \theoremstyle{plain} %% This is the default

\newtheorem{thm}{Theorem}[chapter]
\newtheorem{cor}[thm]{Corollary}
\newtheorem{lem}[thm]{Lemma}
\newtheorem{prop}[thm]{Proposition}
\newtheorem{ax}{Axiom}

\theoremstyle{definition}
\newtheorem{defn}{Definition}[chapter]

\theoremstyle{remark}
\newtheorem{rem}{Remark}[chapter]
\newtheorem*{notation}{Notation}

%\numberwithin{equation}{section}


%%%%%%%%%%%%%%%%%%%%%%%%%%%%%%%%%%%%%%%%%%%%%%%%%%%%%%%%%%%%%%%%%%%%%%
%	Macros.							     %
%%%%%%%%%%%%%%%%%%%%%%%%%%%%%%%%%%%%%%%%%%%%%%%%%%%%%%%%%%%%%%%%%%%%%%
%
%	Here some macros that are needed in this document:


\newcommand{\latexe}{{\LaTeX\kern.125em2%
                      \lower.5ex\hbox{$\varepsilon$}}}

\newcommand{\amslatex}{\AmS-\LaTeX{}}

\chardef\bslash=`\\	% \bslash makes a backslash (in tt fonts)
			%	p. 424, TeXbook

\newcommand{\cn}[1]{\texttt{\bslash #1}}

\makeatletter		% Starts section where @ is considered a letter
			% and thus may be used in commands.
\def\square{\RIfM@\bgroup\else$\bgroup\aftergroup$\fi
  \vcenter{\hrule\hbox{\vrule\@height.6em\kern.6em\vrule}%
                                              \hrule}\egroup}
\makeatother		% Ends sections where @ is considered a letter.
			% Now @ cannot be used in commands.

%\nobibliography*


% These values determine the numbering within the text and Table of Contents.
\setcounter{secnumdepth}{3}    % Number subsections in the chapters.
\setcounter{tocdepth}{2} % include subsections in the Table of Contents