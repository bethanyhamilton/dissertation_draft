
\subsection{Satterthwaite Approximations for Model-Based Degrees of Freedom for the Overall Pooled Effect}

For the model-based hypothesis test:

\begin{equation} \label{model-basedwaldtest}
 t^M = \frac{\hat{\mu} - d}{\sqrt{V^M}}
\end{equation}

Where $V^M = 1/W$. If the null hypothesis, $H_0: \mu = d$ holds, then $t^M$ follows a standard normal distribution, otherwise $t^M$ follows a normal distribution with mean $\lambda = \sqrt{W}(\mu - d)$ and unit variance when CHE is correctly specified and the number of independent studies is large. For a smaller number of studies (less than 40 primary studies), then using a non-central Student-t distribution with Satterthwaite approximation to degrees of freedom, $\zeta$, and with non-centrality parameter $\lambda$ would be another alternative (NOTE: J-1 is a simpler alternative but is a rough approximation to the sampling distribution of the model-based test). The power of the model-based Wald test against a two-sided alternative can be approximated by:

\begin{equation}
    F_t(-c_{\alpha/2,\zeta}  | \zeta, \lambda) +1 - F_t(-c_{\alpha/2,\zeta}  | \zeta, \lambda) 
\end{equation}




For the CHE model, the Satterthwaite degrees of freedom are given by:

\begin{equation}\label{CHEdf}
    \zeta = \frac{st - u^2}{sy^2 + tx^2 - 2uxy}
\end{equation}

where 

\begin{equation}
     x = \frac{1}{W} \sum_{j=1}^J w^2_j
\end{equation}

\begin{equation}
    y = \frac{1}{W}\sum_{j=1}^J \frac{w^2_j}{k_j}
\end{equation}

\begin{equation}
    s = x^2 + Wx - \frac{2}{W}\sum_{j=1}^J w^3_j
\end{equation}

\begin{equation}
    t = y^2 + \sum_{j=1}^J \frac{w^2_j}{k_j^2} + \sum_{j=1}^J \frac{k_j -1}{(\hat{\omega}^2 + (1-\rho)\sigma^2_j )^2} - \frac{2}{W}\sum_{j=1}^J \frac{w^2_j}{k_j^2}
\end{equation}
    
\begin{equation}
    u = xy + Wy - \frac{2}{W}\sum_{j=1}^J\frac{w_j^3}{k_j}
\end{equation}

Goal: To derive equations used in the Satterthwaite degrees of freedom for the model-based variance estimator used in the test for the overall pooled average effect size (intercept). 

Introduction: 

Let $\mu_p$ be the overall average effect size for one category $p$ and $d_p$ be a specific value of effect size. $V_p$ is some variance estimator that can be model-based or robust. $J_p$ is the total number of studies per category. $k_{j_p}$ is effect sizes within each study $j_p$ and $K_p$ is the total number of effect sizes within a category. $\textbf{1}_{j_p}$  is a vector of 1's with size $k_{j_p} \times 1$. $ \textbf{1}$ is a vector of 1's with size of $k_p \times 1$. $\textbf{I}_{j_p}$ is an identity matrix with a size of $k_{j_p} \times k_{j_p}$. $T_{ij_p}$ is effect size estimate $i$ from study $j_p$. $\textbf{T}_{j_p}$ is a vector of effect size estimates from study $j_p$ for $j_p=1,...,J_p$.  





When $V$ is an unbiased estimator of the true sampling variance of $\hat{\mu}$, then the Satterthwaite degrees of freedom are defined as:

\begin{equation}
    \frac{2 \times [E(V)]^2}{Var(V)}
\end{equation}


The CHE data-generating process can be written as:

\begin{equation}
    \mathbf{T}_j = N(\mu\mathbf{1}_j, \mathbf{\Phi}_j)
\end{equation}

where 

\begin{equation}
    \mathbf{\Phi}_j = (\tau^2 + \rho\sigma_j^2)\mathbf{1}_j\textbf{1}_j' + (\omega^2 + (1-\rho)\sigma^2_j)\mathbf{I}_j.
\end{equation}

Let $\mathbf{\Phi} = diag(\mathbf{\Phi_1}, \mathbf{\Phi_2}, ..., \mathbf{\Phi_J})$. Twice the restricted log likelihood of the CHE model is therefore


\begin{equation}\label{LLCHE}
    2\times l_R(\tau^2, \omega^2, \rho) = c - \sum^J_{j=1}log|\mathbf{\Phi_j}| - log(\sum^J_{j=1}\mathbf{1'}_j\mathbf{\Phi}^{-1}_j\mathbf{1}_j)-\mathbf{T}'\mathbf{Q}\mathbf{T},
\end{equation}

where $\mathbf{Q} = \mathbf{\Phi}^{-1}-\mathbf{\Phi}^{-1}\mathbf{1}(\mathbf{1}'\mathbf{\Phi}^{-1}\mathbf{1})^{-1}\mathbf{1}'\mathbf{\Phi}^{-1}$.


\subsubsection{Satterthwaite Approximation for the Model-Based Variance Estimator}

Under the CHE model, the model-based variance estimator is $V^M = 1/W$, where $W$ is $W = \sum^J_{j=1} w_j$ where $w_j$ for CHE is given in Equation \ref{CHEweights}.


They use the delta method to approximate the sampling variance of $V^M$ in terms of the inverse expected information matrix of the restricted likelihood. Let $\mathbf{I}^E$ denote the expected information matrix of the CHE model with entries. 



\begin{equation}
     2I^E = \begin{bmatrix}
                s & u \\
                u & t 
            \end{bmatrix} 
\end{equation}


For the inverse of $\mathbf{I}^E$:

\begin{equation}
     I^E = \begin{bmatrix}
                s/2 & u/2 \\
                u/2 & t/2 
            \end{bmatrix} \nonumber
\end{equation}



\begin{equation}
     (I^E)^{(-1)} = \begin{bmatrix}
                s/2 & u/2 \\
                u/2 & t/2 
            \end{bmatrix}^{(-1)} \nonumber
\end{equation}


Note: Take the inverse of both sides where:  
    $\begin{bmatrix}
                a & b \\
                c & d 
            \end{bmatrix}^{(-1)} = \frac{1}{ad +bc}\begin{bmatrix}
                d & -b \\
                -c & a 
            \end{bmatrix}$ 

\begin{empheq}{align}
        = \frac{1}{(s/2)(t/2)-(u/2)(u/2)}\begin{bmatrix}
                t/2 & -u/2 \\
                -u/2 & s/2 
            \end{bmatrix} \nonumber
\end{empheq}

\begin{empheq}{align}
        = \frac{4}{(s)(t)-(u^2)}\begin{bmatrix}
                t/2 & -u/2 \\
                -u/2 & s/2 
            \end{bmatrix} \nonumber
\end{empheq}


Let $x = W \times \partial V^M/ \partial \hat{\tau}^2$ and $y = W \times \partial V^M/ \partial \hat{\omega}^2$. Therefore, 
they can be rewritten as $x/W = \partial V^M/ \partial \hat{\tau}^2$ and $y/W = \partial V^M/ \partial \hat{\omega}^2$. 

Now the delta method approximation for the variance of $V^M$ is:


\begin{equation}
    Var(V^M) \approx \left(  \frac{\partial V^M}{\partial (\tau^2, \omega^2)} \right)' (\mathbf{I}^E)^{-1} \left(\frac{\partial V^M}{\partial (\tau^2, \omega^2)}\right)
\end{equation}

Note: To rewrite the partial derivative in matrix form is  $\frac{\partial V^M}{\partial (\tau^2, \omega^2)} = \begin{bmatrix}
        \partial V^M/ \partial \omega^2 \\
         \partial V^M/ \partial \tau^2 
    \end{bmatrix}$

Through substitution and expanding the matrix:

\begin{equation}
    =\begin{bmatrix}
        \partial V^M/ \partial \omega^2 &
         \partial V^M/ \partial \tau^2 
    \end{bmatrix} \frac{4}{st-u^2}\begin{bmatrix}
                t/2 & -u/2 \\
                -u/2 & s/2 
            \end{bmatrix}
    \begin{bmatrix}
        \partial V^M/ \partial \omega^2 \\
         \partial V^M/ \partial \tau^2 
    \end{bmatrix}
     \nonumber
\end{equation}


\begin{equation}
    =\begin{bmatrix}
        y/W &
         X/W 
    \end{bmatrix} \frac{4}{st-u^2}\begin{bmatrix}
                t/2 & -u/2 \\
                -u/2 & s/2 
            \end{bmatrix}
    \begin{bmatrix}
        y/W \\
         x/W
    \end{bmatrix}
    \nonumber
\end{equation}

multiply everything out to get: 

\begin{equation}
    Var(V^M) = \frac{2(sy^2+tx^2-2uxy)}{W^2(st-u^2)}
\end{equation}


Now find the degrees if freedom for CHE ( $ \zeta $; Equation \ref{CHEdf} ). If $E(V^M)$ is estimated using $1/W$, the Satterthwaite degrees of freedom for $V^M$ are: 

\begin{equation}
    \zeta =  \frac{2 \times [E(V)]^2}{Var(V)} =  \frac{st - u^2}{sy^2 + tx^2 - 2uxy}
    \nonumber
\end{equation}

since 

\begin{equation}
    \zeta =  \frac{2 \times [E(V)]^2}{Var(V)} = \frac{2[1/W]^2}{Var(V^M}
    \nonumber
\end{equation}

substituting $Var(V^M)$ from above

\begin{equation}
     \zeta = 2\left(\frac{1}{W}\right)^2\left(\frac{W^2(st-u^2)}{2(sy^2+tx^2-2uxy)}\right) =  \frac{st - u^2}{sy^2 + tx^2 - 2uxy}
     \nonumber
\end{equation}


The derivations for x, y, s, t, and u are below. 

\paragraph{Finding x and y}

Below is how to find x and y. Using Equation \ref{CHEweights} for $w_j$, take the partial derivative with respect to $\hat{\tau^2}$:

\begin{equation}
    \begin{split}
    \partial w_j / \partial \hat{\tau}^2 & = (-1) (k_j) (k_j\hat{\tau}^2 +k_j \rho \sigma^2_j + \hat{\omega}^2 + (1- \rho) \sigma^2_j)^{-2}(k_j) \\
    & = \frac{-(k_j)^2}{(k_j\hat{\tau}^2 +k_j \rho \sigma^2_j + \hat{\omega}^2 + (1- \rho) \sigma^2_j)^2} \\
    & = - w_j^2
    \end{split}
    \nonumber
\end{equation}

Now take partial derivative of $w_j$ with respect to $\hat{\omega}^2$:

\begin{equation}
    \begin{split}
    \partial w_j / \partial \hat{\omega}^2 & = (-1) (k_j) (k_j\hat{\tau}^2 +k_j \rho \sigma^2_j + \hat{\omega}^2 + (1- \rho) \sigma^2_j)^{-2} \\
    & = \frac{-(k_j)}{(k_j\hat{\tau}^2 +k_j \rho \sigma^2_j + \hat{\omega}^2 + (1- \rho) \sigma^2_j)^2} \times \frac{k_j}{k_j} \\
    & = \frac{- w_j^2}{k_j}
    \end{split}
    \nonumber
\end{equation}

Since $V^M$ is $1/W$ and $W= \sum_{j=1}^Jw_j$, then:

\begin{equation}
    \begin{split}
        x & = W \frac{\partial V^M}{\partial \hat{\tau}^2} = W \frac{\partial}{\partial \hat{\tau}^2} \left( \sum_{j=1}^Jw_j\right)^{(-1)} \\
         & = W \left( \frac{-1}{ (\sum_{j=1}^J w_j)^2 } \right) \sum_{j=1}^J \frac{\partial w_j}{\partial \hat{\tau}^2} \\
         & = W \left( \frac{-1}{ W^2 } \right) \sum_{j=1}^J - w_j^2 \\
         & = \frac{1}{W} \sum_{j=1}^J w_j^2
    \end{split}
    \nonumber
\end{equation}

Now to find y: 


\begin{equation}
    \begin{split}
        y & = W \frac{\partial V^M}{\partial \hat{\omega}^2} = W \frac{\partial}{\partial \hat{\omega}^2} \left( \sum_{j=1}^Jw_j\right)^{(-1)} \\
         & = W \left( \frac{-1}{ (\sum_{j=1}^J w_j)^2 } \right) \sum_{j=1}^J \frac{\partial w_j}{\partial \hat{\omega}^2} \\
         & = W \left( \frac{-1}{ W^2 } \right) \sum_{j=1}^J \frac{-w_j^2}{k_j} \\
         & = \frac{1}{W} \sum_{j=1}^J \frac{w_j^2}{k_j}
    \end{split}
    \nonumber
\end{equation}

\paragraph{Finding s, t, u}

Using the $K\times X$ matrix $\mathbf{Q}$,  $\mathbf{Q} = \mathbf{\Phi}^{-1}-\mathbf{\Phi}^{-1}\mathbf{1}(\mathbf{1}'\mathbf{\Phi}^{-1}\mathbf{1})^{-1}\mathbf{1}'\mathbf{\Phi}^{-1}$, Giesbrecht and Burns (1985) gave an expression for the expected information and the derivative of $\mathbf{\Phi}$ with respect to the variance components. For the CHE model: 

\begin{equation}
    \begin{split}
        s & = tr(\mathbf{Q} \frac{\partial \mathbf{\Phi}}{\partial \hat{\tau}^2}\mathbf{Q}\frac{\partial \mathbf{\Phi}}{\partial \hat{\tau}^2}) \\
        t & = tr(\mathbf{Q} \frac{\partial \mathbf{\Phi}}{\partial \hat{\omega}^2}\mathbf{Q}\frac{\partial \mathbf{\Phi}}{\partial \hat{\omega}^2}) \\
        u & = tr(\mathbf{Q} \frac{\partial \mathbf{\Phi}}{\partial \hat{\tau}^2}\mathbf{Q}\frac{\partial \mathbf{\Phi}}{\partial \hat{\omega}^2}) 
    \end{split}
\end{equation}

For $\frac{\partial \mathbf{\Phi}}{\partial \hat{\tau}^2}$ it is equal to $\mathbf{Z}\mathbf{Z}'$ which is a $K \times J$ block diagonal matrix of $\mathbf{1}_j$, $Z= diag(\mathbf{1}_1, \mathbf{1}_2, ..., \mathbf{1}_J)$. Also, $\frac{\partial \mathbf{\Phi}}{\partial \hat{\omega}^2}$ is a $K \times K$ identity matrix, $\mathbf{I}$. Using the properties of trace and substitution, it follows that:

\begin{equation}
    \begin{split}
        s & = tr(\mathbf{Q}\mathbf{Z}\mathbf{Z}'\mathbf{Q}\mathbf{Z}\mathbf{Z}') = tr(\mathbf{Z}'\mathbf{Q}\mathbf{Z}\mathbf{Z}'\mathbf{Q}\mathbf{Z}) \\
        t & = tr(\mathbf{Q} \mathbf{Q}) \\
        u & = tr(\mathbf{Q} \mathbf{Z}\mathbf{Z}'\mathbf{Q}) = tr(\mathbf{Z}'\mathbf{Q}\mathbf{Q}\mathbf{Z})
    \end{split}
\end{equation}



So, now we need to find $\mathbf{Q}$. Therefore, we first need to find $\mathbf{\Phi}^-1$ or more specifically, $\mathbf{\Phi}_j^-1$. This is because,  $\mathbf{\Phi}$ is a $K \times K$ block diagonal with the $\mathbf{\Phi}_j$ along the diagonal. The inverse of $\mathbf{\Phi}$ is just the a $K \times K$ black diagonal with the $\mathbf{\Phi}_j^-1$ along the diagonal. 

To find the inverse of $\mathbf{\Phi}_j$, we can use the Woodbury matrix identity or a lemma from (Miller, 1981). 

If $A$ and $A+B$ are invertible and $B$ has rank 1, then let $g = trace(BA^{-1})$. Then $g\neq -1$ and $(A +B)^{-1} = A^{-1} - \frac{1}{1+g}A^{-1}BA^{-1}$.

Let $A = (\omega^2 + (1- \rho)\sigma^2_j)\mathbf{I}_j$ and $B=(\tau^2 + \rho\sigma_j^2)\mathbf{1}_j\mathbf{1}_j'$. B has a rank 1. $A^{-1} = \frac{\mathbf{I}_j}{(\omega^2 +(1- \rho)\sigma_j^2)}$. For the g: 

\begin{equation}
    \begin{split}
        g &= tr(BA^{-1}) \\
        & = tr\left(\frac{(\tau^2 + \rho\sigma_j^2)}{\omega^2 +(1-\rho)\sigma_j^2} \mathbf{1}_j\mathbf{1}_j' \mathbf{I}_j    \right) \\
        & = \frac{(\tau^2 + \rho\sigma_j^2)}{\omega^2 +(1-\rho)\sigma_j^2} tr( \mathbf{1}_j\mathbf{1}_j') \\
        & = \frac{(\tau^2 + \rho\sigma_j^2)}{\omega^2 +(1-\rho)\sigma_j^2} k_j
    \end{split}
    \nonumber
\end{equation}

Now for the $\mathbf{\Phi}_j^-1$:


\begin{equation}
    \begin{split}
        \mathbf{\Phi}_j^{-1} & = A^{-1} - \frac{1}{1+g}A^{-1}BA^{-1} \\
        & = A^{-1}(\mathbf{I}_j - \frac{1}{1+g}BA^{-1}) \\
        & = \frac{1}{\omega^2 +(1-\rho)\sigma^2_j}(\mathbf{I}_j - \frac{1}{1+\frac{k_j(\tau^2 + \rho\sigma^2_j)}{\omega^2 +(1-\rho)\sigma^2_j}}\frac{(\tau^2 + \rho\sigma^2_j)}{\omega^2 +(1-\rho)\sigma^2_j}\mathbf{1}_j\mathbf{1}_j')  \\
        & = \frac{1}{\omega^2 +(1-\rho)\sigma^2_j}(\mathbf{I}_j - \frac{\tau^2 + \rho\sigma^2_j}{\omega^2 +(1-\rho)\sigma^2_j+k_j(\tau^2 + \rho\sigma^2_j)}\mathbf{1}_j\mathbf{1}_j') 
    \end{split}
\end{equation}

Now the next terms in $\mathbf{Q}$ in terms of $\Phi_j$ are $\mathbf{\Phi}_j^{-1}\mathbf{1}_j$,  $\mathbf{1}_j'\mathbf{\Phi}_j^{-1}\mathbf{1}_j$, and $\mathbf{1}_j'\mathbf{\Phi}_j^{-1}$. The values for those are: $\mathbf{\Phi}_j^{-1}\mathbf{1}_j = \frac{w_j}{k_j}\mathbf{1}_j $,  $\mathbf{1}_j'\mathbf{\Phi}_j^{-1}\mathbf{1}_j = w_j$, and $\mathbf{1}_j'\mathbf{\Phi}_j^{-1} = \mathbf{1}_j'\frac{w_j}{k_j}$. Also, note, $tr(\mathbf{1}'\mathbf{1})= K$


%%%%%%%%%%%%%%%%%%%%%%%%%%%%%%%%%%%%%%%%%%%%%%%%%%%%%%%%%%%%%%%%%%%%%%%%%%%%%%%%%%%%%%%%%%%%%%%%%%%%%%%%%%%
% s
%%%%%%%%%%%%%%%%%%%%%%%%%%%%%%%%%%%%%%%%%%%%%%%%%%%%%%%%%%%%%%%%%%%%%%%%%%%%%%%%%%%%%%%%%%%%%%%%%%%%%%%%%%%

Now for s:

\begin{equation}
    s = tr(\mathbf{Z}'\mathbf{Q}\mathbf{Z}\mathbf{Z}'\mathbf{Q}\mathbf{Z}) 
    \nonumber
\end{equation}


For $\mathbf{Z}'\mathbf{Q}\mathbf{Z}$:


\begin{equation}
   = \begin{bmatrix}
        \mathbf{1}_1' & & & \\
         & \mathbf{1}_2' & &\\
         & & \ddots &  \\
         & &  & \mathbf{1}_J'
    \end{bmatrix}_{J \times K} \times (\mathbf{\Phi}^{-1}- \mathbf{\Phi}^{-1}\mathbf{1}(\mathbf{1}'\mathbf{\Phi}^{-1}\mathbf{1})^{-1}\mathbf{1}'\mathbf{\Phi}^{-1}) \times
   \begin{bmatrix}
        \mathbf{1}_1 & & & \\
         & \mathbf{1}_2 & &\\
         & & \ddots &  \\
         & &  & \mathbf{1}_J
    \end{bmatrix}_{K \times J}
    \nonumber
\end{equation}


For one diagonal:

\begin{equation}
    \begin{split}
         & = \left(\mathbf{1}'_j\mathbf{\Phi_j}^{-1}-\mathbf{1}'_j\left(\frac{w_j}{k_j}\mathbf{1}_j\right)\left(\frac{1}{wj}\right)\left(\mathbf{1}'\frac{w_j}{k_j}\right)\right)\mathbf{1}_j \\
         & = \left(\mathbf{1}'_j\mathbf{\Phi_j}^{-1}-\left(\mathbf{1}'_j\frac{w_j}{k_j}\mathbf{1}_j\right)\left(\frac{1}{wj}\right)\left(\mathbf{1}'\frac{w_j}{k_j}\right)\right)\mathbf{1}_j \\
         & = \left(\mathbf{1}'_j\mathbf{\Phi_j}^{-1}-\left(w_j\right)\left(\frac{1}{wj}\right)\left(\mathbf{1}'\frac{w_j}{k_j}\right)\right)\mathbf{1}_j \\
         & = \left(\mathbf{1}'_j\mathbf{\Phi_j}^{-1}-\left(\frac{1}{wj}\right)\left(w_j\right)\left(\mathbf{1}'\frac{w_j}{k_j}\right)\right)\mathbf{1}_j \\
         & = \left(\mathbf{1}'_j\mathbf{\Phi_j}^{-1}\mathbf{1}_j -\left(\frac{1}{wj}\right)\left(w_j\right)\left(\mathbf{1}'\frac{w_j}{k_j}\mathbf{1}_j \right)\right)\\
         & = \left(w_j -\left(\frac{1}{wj}\right)\left(w_j\right)\left(w_j \right)\right)\\
         & = \left(w_j -\left(\frac{1}{wj}\right)\left( w_j^2 \right)\right)\\
    \end{split}
    \nonumber
\end{equation}
written as matrix form:
\begin{equation}
     \begin{bmatrix}
        w_1 & &   \\
         &  \ddots &  \\
         & &   w_J
    \end{bmatrix} - \frac{1}{W}
     \begin{bmatrix}
         w_1^2 & \dots & w_1w_j  \\
         \vdots & \ddots & \vdots \\
         w_jw_1 & \dots &  w_J^2
    \end{bmatrix} 
    \nonumber
\end{equation}
Let,
\begin{equation}
     A = \begin{bmatrix}
        w_1 & &   \\
         &  \ddots &  \\
         & &   w_J
    \end{bmatrix} 
    \nonumber
\end{equation}
\begin{equation}
     B = \begin{bmatrix}
         w_1^2 & \dots & w_1w_j  \\
         \vdots & \ddots & \vdots \\
         w_jw_1 & \dots &  w_J^2
    \end{bmatrix}  
    \nonumber
\end{equation}

\begin{equation}
\begin{split}
    (\mathbf{Z}'\mathbf{Q}\mathbf{Z})(\mathbf{Z}'\mathbf{Q}\mathbf{Z}) & = (A-B)^2 \\
     & = A^2 + B^2 - 2AB
\end{split}
     \nonumber
\end{equation}

\begin{equation}
\begin{split}
     =\begin{bmatrix}
        w_1^2 & &   \\
         &  \ddots &  \\
         & &   w_J^2
    \end{bmatrix}  & + \frac{1}{W^2}
    \begin{bmatrix}
         w_1^2 & \dots & w_1w_j  \\
         \vdots & \ddots & \vdots \\
         w_jw_1 & \dots &  w_J^2
    \end{bmatrix} 
     \begin{bmatrix}
         w_1^2 & \dots & w_1w_j  \\
         \vdots & \ddots & \vdots \\
         w_jw_1 & \dots &  w_J^2
    \end{bmatrix} \\ &- 
    \frac{2}{W} \begin{bmatrix}
        w_1 & &   \\
         &  \ddots &  \\
         & &   w_J
    \end{bmatrix}  \begin{bmatrix}
         w_1^2 & \dots & w_1w_j  \\
         \vdots & \ddots & \vdots \\
         w_jw_1 & \dots &  w_J^2
    \end{bmatrix}
\end{split}
    \nonumber
\end{equation}

\begin{equation}
\begin{split}
     = & \begin{bmatrix}
        w_1^2 & &   \\
         &  \ddots &  \\
         & &   w_J^2
    \end{bmatrix}   + \frac{1}{W^2}   \times \\
     & \begin{bmatrix}
         w^4_1 +w^2_1w^2_2 + ...+w^2_1w^2_j & & \\
         & w^4_2 +w^2_2w^2_1 + ...+w^2_2w^2_j & \\
          & &  \ddots &  \\
         & &  &w^4_j +w^2_1w^2_j + ...+w^2_{j-1}w^2_j 
     \end{bmatrix} \\ &- 
    \frac{2}{W} \begin{bmatrix}
        w_1^3 & &   \\
         &  \ddots &  \\
         & &   w_J^3
    \end{bmatrix} 
\end{split}
    \nonumber
\end{equation}


Now using properties of trace, $tr( \mathbf{Z}'\mathbf{Q}\mathbf{Z}\mathbf{Z}'\mathbf{Q}\mathbf{Z})$,

\begin{equation}
    =\sum_{j=1}^J w_j^2 + \frac{1}{W^2} (w^4_1+w^2_1w^2_2+...+w^4_2+w^2_2w^2_1 +...) - \frac{2}{W} \sum_{j=1}^Jw_j^3
    \nonumber
    \\
\end{equation}


For the middle term, $1/W^2(w^4_1+w^2_1w^2_2+...+w^4_2+w^2_2w^2_1 +...)$, factor out $w^2$ from each diagonal cell to become: $1/W^2(w^2_1(w^2_1+w^2_2+ ... + w_j^2)+...+w_2^2(w^2_2+w^2_1+ ...+ w_j^2 )+...)$ this becomes $1/W^2(\sum_{j=1}^Jw_j^2\sum_{j=1}^Jw_j^2) = 1/W^2(\sum_{j=1}^Jw_j^2)^2 = x^2$. The first term, $\sum_{j=1}^J w_j^2 $ is $Wx$, so therefore $s$ is equal to:

\begin{equation}
    s = x^2 + Wx - \frac{2}{W} \sum_{j=1}^Jw_j^3
    \nonumber
\end{equation}

%%%%%%%%%%%%%%%%%%%%%%%%%%%%%%%%%%%%%%%%%%%%%%%%%%%%%%%%%%%%%%%%%%%%%%%%%%%%%%%%%%%%%%%%%%%%%%%%%%%%%%%%%%%
% u
%%%%%%%%%%%%%%%%%%%%%%%%%%%%%%%%%%%%%%%%%%%%%%%%%%%%%%%%%%%%%%%%%%%%%%%%%%%%%%%%%%%%%%%%%%%%%%%%%%%%%%%%%%%

Now for u: 

\begin{equation}
    u = tr(\mathbf{Z}'\mathbf{Q}\mathbf{Q}\mathbf{Z}) 
    \nonumber
\end{equation}



For $\mathbf{Z}'\mathbf{Q}$:


\begin{equation}
  \mathbf{Z}'\mathbf{Q} = \begin{bmatrix}
        \mathbf{1}_1' & & & \\
         & \mathbf{1}_2' & &\\
         & & \ddots &  \\
         & &  & \mathbf{1}_j'
    \end{bmatrix}_{J \times K} \times (\mathbf{\Phi}^{-1}- \mathbf{\Phi}^{-1}\mathbf{1}(\mathbf{1}'\mathbf{\Phi}^{-1}\mathbf{1})^{-1}\mathbf{1}'\mathbf{\Phi}^{-1}) 
    \nonumber
\end{equation}


\begin{equation}
   = \begin{bmatrix}
        \mathbf{1}_1'\mathbf{\phi}_1^{-1} & & & \\
         & \mathbf{1}_2'\mathbf{\phi}_2^{-1} & &\\
         & & \ddots &  \\
         & &  & \mathbf{1}_j'\mathbf{\phi}_j^{-1}
    \end{bmatrix}_{J \times K}  - \begin{bmatrix}
        w_1 \\
        w_2 \\
        \vdots \\
        w_j
    \end{bmatrix} \frac{1}{W} 
    \begin{bmatrix}
        \frac{w_1}{k_1}\mathbf{1}_1' & \frac{w_2}{k_2}\mathbf{1}_2' & . & . & . & \frac{w_j}{k_j}\mathbf{1}_j'
    \end{bmatrix}
    \nonumber
\end{equation}


\begin{equation}
   = \begin{bmatrix}
        \mathbf{1}_1'\frac{w_1}{k_1} & &  \\
         & \mathbf{1}_2'\frac{w_2}{k_2} &  \\
         & & \ddots &  \\

         & & &   \mathbf{1}_j'\frac{w_j}{k_j}
    \end{bmatrix} -  \frac{1}{W} 
      \begin{bmatrix}
         \frac{w_1^2}{k_1}\mathbf{1}_1' & \frac{w_1w_2}{k_2}\mathbf{1}_2' &  \dots & \frac{ w_1w_j}{k_j}\mathbf{1}_j'  \\
         &  \frac{w_2^2}{k_2}\mathbf{1}_2' &  &\\
         & & \ddots &  \\

         & & & \frac{w_j^2}{k_j}\mathbf{1}_j'
    \end{bmatrix} 
    \nonumber
\end{equation}

\begin{equation}
  \mathbf{Z}'\mathbf{Q}  = \begin{bmatrix}
        \frac{w_1}{k_1}(1 - \frac{w_1}{W})\mathbf{1}_1' & -\frac{w_1w_2}{Wk_2}\mathbf{1}_2' &  \dots  \\
         &  \frac{w_2}{k_2}(1 - \frac{w_2}{W})\mathbf{1}_2' & &  \\
         & & \ddots &  \\
         & & & \frac{w_j}{k_j}(1 - \frac{w_j}{W})\mathbf{1}_j'
    \end{bmatrix} 
    \nonumber
\end{equation}

Now for $\mathbf{Q}\mathbf{Z}$:

\begin{equation}
  \begin{split}
      = & \begin{bmatrix}
        \mathbf{\phi}_1^{-1} &  & &\\
         & \mathbf{\phi}_2^{-1} &  &\\
         & & \ddots &  \\
         & & &   \mathbf{\phi}_j^{-1}
    \end{bmatrix}\begin{bmatrix}
        \mathbf{1}_1 & & &  \\
         & \mathbf{1}_2 & &  \\
         & & \ddots &  \\
         & & &  \mathbf{1}_j
    \end{bmatrix} -  \frac{1}{W} \times \\ 
    & \begin{bmatrix}
        (\frac{w_1}{k_1})^2\mathbf{1}_1\mathbf{1}_1' & \frac{w_1w_2}{k_1k_2}\mathbf{1}_1\mathbf{1}_2' &  \dots & &  \\
         &  (\frac{w_2}{k_2})^2\mathbf{1}_2\mathbf{1}_2' & & &   \\
         & & \ddots & &  \\
         & & & &  (\frac{w_j}{k_j})^2\mathbf{1}_j\mathbf{1}_j'
    \end{bmatrix} \begin{bmatrix}
        \mathbf{1}_1 & & &  \\
         & \mathbf{1}_2 & &  \\
         & & \ddots &  \\
         & & &  \mathbf{1}_j
    \end{bmatrix}
  \end{split}
    \nonumber
\end{equation}

\begin{equation}
     = \begin{bmatrix}
        \mathbf{\phi}_1^{-1}\mathbf{1}_1  & & &  \\
         & \mathbf{\phi}_2^{-1}\mathbf{1}_2  & &  \\
         & & \ddots &  \\
         & & & &   \mathbf{\phi}_J^{-1}\mathbf{1}_j 
    \end{bmatrix} - \frac{1}{W} 
    \begin{bmatrix}
        (\frac{w_1}{k_1})^2\mathbf{1}_1\mathbf{1}_1\mathbf{1}_1' & & & &  \\
         &  (\frac{w_2}{k_2})^2\mathbf{1}_2\mathbf{1}_2\mathbf{1}_2' & & &   \\
         & & \ddots & &  \\
         & & & &  (\frac{w_j}{k_j})^2\mathbf{1}_j\mathbf{1}_j\mathbf{1}_j'
    \end{bmatrix}
     \nonumber
\end{equation}

Note: trace of  $\mathbf{1}_j\mathbf{1}_j'$ is $k_j$. 

\begin{equation}
     = \begin{bmatrix}
        \frac{w_1}{k_1}\mathbf{1}_1  & & & &  \\
         & \frac{w_2}{k_2}\mathbf{1}_2  & & &  \\
         & & \ddots & & \\
         & & & &   \frac{w_j}{k_j}\mathbf{1}_j 
    \end{bmatrix} - \frac{1}{W} 
    \begin{bmatrix}
        (\frac{w_1^2}{k_1})\mathbf{1}_1 & &   &  &     \\
         &  (\frac{w_2^2}{k_2})\mathbf{1}_2 & & &   \\
         & & \ddots & &  \\
         & & & &  (\frac{w_J^2}{k_J})\mathbf{1}_j
    \end{bmatrix}
    \nonumber
\end{equation}


\begin{equation}
   \mathbf{Q}\mathbf{Z} =  \begin{bmatrix}
        \frac{w_1}{k_1}(1 - \frac{w_1}{W})\mathbf{1}_1 & &   &  &     \\
         &  \frac{w_2}{k_2}(1 - \frac{w_2}{W})\mathbf{1}_2 & & &   \\
         & & \ddots & &  \\
         & & & & \frac{w_J}{k_J}(1 - \frac{w_J}{W})\mathbf{1}_j
    \end{bmatrix} 
    \nonumber
    \\
\end{equation}


Now $\mathbf{Z}'\mathbf{Q}\mathbf{Q}\mathbf{Z}$ is equal to: 

\begin{equation}
\begin{split}
   = & \begin{bmatrix}
        \frac{w_1}{k_1}(1 - \frac{w_1}{W})\mathbf{1}_1' & -\frac{w_1w_2}{Wk_2}\mathbf{1}_2' &  \dots  &  &  &   \\
         &  \frac{w_2}{k_2}(1 - \frac{w_2}{W})\mathbf{1}_2' & & &   \\
         & & \ddots & &  \\
         & & & &  \frac{w_j}{k_j}(1 - \frac{w_j}{W})\mathbf{1}_j'
    \end{bmatrix}  \times \\
    & \begin{bmatrix}
        \frac{w_1}{k_1}(1 - \frac{w_1}{W})\mathbf{1}_1 & &   &  &    \\
         &  \frac{w_2}{k_2}(1 - \frac{w_2}{W})\mathbf{1}_2 & & &   \\
         & & \ddots & &  \\
         & & & &  \frac{w_j}{k_j}(1 - \frac{w_j}{W})\mathbf{1}_j
    \end{bmatrix} 
\end{split}
    \nonumber
\end{equation}


\begin{equation}
   = \begin{bmatrix}
        (\frac{w_1}{k_1}(1 - \frac{w_1}{W}))^2\mathbf{1}_1'\mathbf{1}_1 & &   &  &     \\
         &  (\frac{w_2}{k_2}(1 - \frac{w_2}{W}))^2\mathbf{1}_2'\mathbf{1}_2 & & &   \\
         & & \ddots & &  \\
         & & & &  (\frac{w_j}{k_j}(1 - \frac{w_j}{W}))^2\mathbf{1}_j'\mathbf{1}_j
    \end{bmatrix} 
    \nonumber
\end{equation}

Now with the properties of trace, to find $u = tr(\mathbf{Z}'\mathbf{Q}\mathbf{Q}\mathbf{Z})$:

\begin{equation}
    \begin{split}
        & = tr\left( \left(\frac{w_1}{k_1}(1 - \frac{w_1}{W})\right)^2\mathbf{1}_1'\mathbf{1}_1  + ... + (\frac{w_j}{k_j}(1 - \frac{w_j}{W}))^2\mathbf{1}_j'\mathbf{1}_j \right) \\
         & = tr\left( \left(\frac{w_1}{k_1}(1 - \frac{w_1}{W})\right)^2k_1 + ... + (\frac{w_j}{k_j}(1 - \frac{w_j}{W}))^2k_j\right) \\
        & = tr\left(\frac{w_1^2}{k_1}(1 - \frac{w_1}{W})^2  + ... + \frac{w_j^2}{k_j}(1 - \frac{w_j}{W})^2 \right) \\
        & = tr\left(\frac{w_1^2}{k_1}(1 + \frac{w_1^2}{W^2}- \frac{2w_1}{W})  + ... + \frac{w_j^2}{k_j}(1 + \frac{w_j^2}{W^2} - \frac{2w_j}{W}) \right) \\
        &= \sum_{j=1}^J \frac{w_j^2}{k_j} + \sum_{j=1}^J \frac{w_j^2}{k_j}\frac{w_j^2}{W^2} - \sum_{j=1}^J \frac{2}{W}\frac{w_j^3}{k_j} \\
        &= \sum_{j=1}^J \frac{w_j^2}{k_j} + \sum_{j=1}^J \frac{w_j^2}{k_j}\frac{w_j^2}{W^2} -  \frac{2}{W}\sum_{j=1}^J\frac{w_j^3}{k_j} \\
         &= Wy  -  \frac{2}{W}\sum_{j=1}^J\frac{w_j^3}{k_j} + \sum_{j=1}^J \frac{w_j^2}{k_j}\frac{w_j^2}{W^2}  \\
         &= Wy  -  \frac{2}{W}\sum_{j=1}^J\frac{w_j^3}{k_j} + \sum_{j=1}^J \frac{1}{W} \frac{w_j^2}{k_j}\frac{1}{W}w_j^2  \\
         &= Wy  -  \frac{2}{W}\sum_{j=1}^J\frac{w_j^3}{k_j} + \sum_{j=1}^J \frac{1}{W} \frac{w_j^2}{k_j }\sum_{j=1}^J\frac{1}{W}w_j^2  \\
         &= Wy  -  \frac{2}{W}\sum_{j=1}^J\frac{w_j^3}{k_j} + xy
    \end{split}
    \nonumber
\end{equation}

%%%%%%%%%%%%%%%%%%%%%%%%%%%%%%%%%%%%%%%%%%%%%%%%%%%%%%%%%%%%%%%%%%%%%%%%%%%%%%%%%%%%%%%%%%%%%%%%%%%%%%%%%%%
% t
%%%%%%%%%%%%%%%%%%%%%%%%%%%%%%%%%%%%%%%%%%%%%%%%%%%%%%%%%%%%%%%%%%%%%%%%%%%%%%%%%%%%%%%%%%%%%%%%%%%%%%%%%%%
Now for t: 

\begin{equation}
    t = tr(\mathbf{Q}\mathbf{Q}) 
    \nonumber
\end{equation}

\begin{equation}
    \begin{split}
       QQ & = (\mathbf{\Phi}^{-1} - \mathbf{\Phi}^{-1}\mathbf{1}(\mathbf{1}'\mathbf{\Phi}^{-1}\mathbf{1})^{-1}\mathbf{1}'\mathbf{\Phi}^{-1})^2 \\
       & = \left(\mathbf{\Phi}^{-1} - \frac{1}{W} 
       \begin{bmatrix} 
       \frac{w_1}{k_1}\mathbf{1}_1 \\
       \vdots \\
       \frac{w_j}{k_j}\mathbf{1}_j
       \end{bmatrix} \begin{bmatrix}
           \frac{w_1}{k_1}\mathbf{1}_1' & \dots & \frac{w_j}{k_j}\mathbf{1}_j'
       \end{bmatrix}  \right)^2
    \end{split}
    \nonumber
\end{equation}

Let $A = \mathbf{\Phi}^{-1}$ and $B = \frac{1}{W} 
       \begin{bmatrix} 
       \frac{w_1}{k_1}\mathbf{1}_1 \\
       \vdots \\
       \frac{w_j}{k_j}\mathbf{1}_j
       \end{bmatrix} \begin{bmatrix}
           \frac{w_1}{k_1}\mathbf{1}_1' & \dots & \frac{w_j}{k_j}\mathbf{1}_j'
       \end{bmatrix}$.

By substituting $A$ and $B$, we get:

\begin{equation}
    \begin{split}
        QQ & = (A - B)^2 = A^2 + B^2 - 2AB \\
        tr(QQ) & = tr(A^2) + tr(B^2) - tr(2AB)
    \end{split}   
     \nonumber
\end{equation}

Now we need to find $tr(A^2)$, $ tr(B^2)$, and $tr(2AB)$.

For $A^2$:


\begin{equation}
    \begin{split}
        (\mathbf{\Phi}^{-1})^2 & = \begin{bmatrix}
        (\mathbf{\Phi}_1^{-1})^2 & &  \\
        & \ddots &  \\
        & & (\mathbf{\Phi}_j^{-1})^2 
    \end{bmatrix} \\
       (\mathbf{\Phi}_j^{-1})^2  & = \left( \frac{1}{\omega^2 + (1-\rho) \sigma^2_j} \left(\mathbf{I}_j - \frac{\tau^2 +\rho\sigma_j^2}{k_j(\tau^2 + \rho\sigma^2_j)+\omega^2+(1-\rho)\sigma_j^2} \mathbf{1}_j \mathbf{1}_j'  \right) \right)^2 
    \end{split}
     \nonumber
\end{equation}


Let $n_j = \omega^2 + (1-\rho) \sigma^2_j$ and $m_j = \frac{\tau^2 +\rho\sigma_j^2}{k_j(\tau^2 + \rho\sigma^2_j)+\omega^2+(1-\rho)\sigma_j^2}$. Then:

\begin{equation}
    \begin{split}
         (\mathbf{\Phi}_j^{-1})^2 & = \frac{1}{n_j^2} \left( \mathbf{I}^2_j + m^2_j \mathbf{1}_j \mathbf{1}_j' \mathbf{1}_j \mathbf{1}_j'  - 2m_j\mathbf{1}_j \mathbf{1}_j'  \right) \\
         & = \frac{1}{n_j^2} \left( \mathbf{I}^2_j + m^2_j \begin{bmatrix}
             k_j & \dots & k_j \\
             \vdots & \ddots  & \vdots  \\
             k_j &  \dots  &   k_j
         \end{bmatrix}_{k_j \times k_j}  - \begin{bmatrix}
             2m_j & \dots & 2m_j \\
             \vdots & \ddots  & \vdots  \\
             2m_j &  \dots  &   2m_j
         \end{bmatrix}_{k_j \times k_j}  \right) \\
         & = \frac{1}{n_j^2} \left( \begin{bmatrix}
             m_j^2 k_j - 2m_j + 1 & m_j^2 k_j - 2m_j & \dots & m_j^2 k_j - 2m_j \\
             m_j^2 k_j - 2m_j  & m_j^2 k_j - 2m_j +1 & \dots & m_j^2 k_j - 2m_j  \\
             \vdots  & \vdots & \ddots &  \vdots \\
             m_j^2 k_j - 2m_j  & m_j^2 k_j - 2m_j  & \dots&  m_j^2 k_j - 2m_j +1 \\
         \end{bmatrix}  \right) \\ 
         & = \left( \begin{bmatrix}
             \frac{m_j^2 k_j - 2m_j + 1}{n_j^2} & \frac{m_j^2 k_j - 2m_j}{n_j^2}& \dots & \frac{m_j^2 k_j - 2m_j}{n_j^2} \\
             \frac{m_j^2 k_j - 2m_j}{n_j^2} & \frac{m_j^2 k_j - 2m_j + 1}{n_j^2} & \dots & \frac{m_j^2 k_j - 2m_j}{n_j^2}  \\
             \vdots  & \vdots & \ddots &  \vdots \\
             \frac{m_j^2 k_j - 2m_j}{n_j^2} & \frac{m_j^2 k_j - 2m_j}{n_j^2} & \dots& \frac{m_j^2 k_j - 2m_j + 1}{n_j^2}\\
         \end{bmatrix}  \right) 
    \end{split}
    \nonumber
\end{equation}

Now take $tr(A^2$):

\begin{equation}
    \begin{split}
        tr(A^2) & = \sum_{j=1}^J \frac{m_j^2 k_j^2 - 2m_jk_j + k_j}{n_j^2} \\
                & = \sum_{j=1}^J \frac{m_j^2 k_j^2 +1 - 2m_jk_j + k_j -1}{n_j^2} \\
                & = \sum_{j=1}^J \frac{(m_jk_j-1)^2+(k_j-1)}{n_j^2} \\
                & = \sum_{j=1}^J \frac{(m_jk_j-1)^2}{n_j^2} + \frac{(k_j-1)}{n_j^2} \\
                & = \sum_{j=1}^J \left(\frac{(m_jk_j-1)}{n_j}\right)^2 + \frac{(k_j-1)}{n_j^2} \\
    \end{split}
    \nonumber
\end{equation}


Now substitute $m_j$ and $n_j$ back in:

\begin{equation}
    \begin{split}
         tr(A^2) & = \sum_{j=1}^J\left( \frac{\frac{(\tau^2 +\rho\sigma_j^2)k_j}{k_j(\tau^2 + \rho\sigma^2_j)+\omega^2+(1-\rho)\sigma_j^2}- 1}{\omega^2 + (1 - \rho)\sigma_j^2} \right)^2 + \frac{k_j-1}{(\omega^2 + (1 - \rho)\sigma_j^2)^2} \\
       & = \sum_{j=1}^J\left( \frac{1}{k_j(\tau^2 + \rho\sigma^2_j)+\omega^2+(1-\rho)\sigma_j^2} \right)^2 + \frac{k_j-1}{(\omega^2 + (1 - \rho)\sigma_j^2)^2} \\
       & = \sum_{j=1}^J\left( \frac{k_j}{k_j} \times \frac{1}{k_j(\tau^2 + \rho\sigma^2_j)+\omega^2+(1-\rho)\sigma_j^2} \right)^2 + \frac{k_j-1}{(\omega^2 + (1 - \rho)\sigma_j^2)^2} \\
       & = \sum_{j=1}^J\left( \frac{1}{k_j} \times w_j \right)^2 + \frac{k_j-1}{(\omega^2 + (1 - \rho)\sigma_j^2)^2} \\
       & = \sum_{j=1}^J\frac{w_j^2}{k_j^2}  + \frac{k_j-1}{(\omega^2 + (1 - \rho)\sigma_j^2)^2} \\
    \end{split}
    \nonumber
\end{equation}

Now that we have found $tr(A^2)$, we need to find $tr(B^2)$ and $tr(2AB)$. 

For $B^2$:

\begin{equation}
    \begin{split}
        B^2 & = \frac{1}{W^2}\begin{bmatrix}
            \frac{w_1}{k_1}\mathbf{1}_1 \\
            \vdots \\
            \frac{w_j}{k_j}\mathbf{1}_j
        \end{bmatrix} \begin{bmatrix}
            \frac{w_1}{k_1}\mathbf{1}_1' & \dots & \frac{w_j}{k_j}\mathbf{1}_j'  
        \end{bmatrix}\begin{bmatrix}
            \frac{w_1}{k_1}\mathbf{1}_1 \\
            \vdots \\
            \frac{w_j}{k_j}\mathbf{1}_j
        \end{bmatrix} \begin{bmatrix}
            \frac{w_1}{k_1}\mathbf{1}_1' & \dots & \frac{w_j}{k_j}\mathbf{1}_j'  
        \end{bmatrix} \\
        & = \frac{1}{W^2}\begin{bmatrix}
             \frac{w_1^2}{k_1^2}\mathbf{1}_1\mathbf{1}_1'& \dots & &\frac{w_1w_j}{k_1k_j}\mathbf{1}_1\mathbf{1}_j' \\
             & \ddots & & &\\
             & & &  \frac{w_j^2}{k_j^2}\mathbf{1}_j\mathbf{1}_j'
        \end{bmatrix} \begin{bmatrix}
             \frac{w_1^2}{k_1^2}\mathbf{1}_1\mathbf{1}_1'& \dots & &\frac{w_1w_j}{k_1k_j}\mathbf{1}_1\mathbf{1}_j' \\
             & \ddots & & &\\
             & & &  \frac{w_j^2}{k_j^2}\mathbf{1}_j\mathbf{1}_j'
        \end{bmatrix} \\
        = & \frac{1}{W^2}\begin{bmatrix}
            \left[\frac{w_1^4}{k_1^4} + \left( \frac{w_1w_2}{k_1k_2} \right)^2 + \dots + \left( \frac{w_1w_j}{k_1k_j} \right)^2  \right] k_1^2 & & \\
            & \ddots & \\
            & & \left[\frac{w_j^4}{k_j^4} +  \dots + \left( \frac{w_jw_1}{k_jk_1} \right)^2  \right] k_j^2
        \end{bmatrix}
    \end{split}
    \nonumber
\end{equation}

Now taking the $tr(B^2)$:

\begin{equation}
    \begin{split}
        tr(B^2) & =  \sum_{j=1}^J\left(\frac{w_1^4}{k_1^4} + \left( \frac{w_1w_2}{k_1k_2} \right)^2 + \dots + \left( \frac{w_1w_j}{k_1k_j} \right)^2  \right) k_1^2 +  \dots +  \left(\frac{w_j^4}{k_j^4} +  \dots + \left( \frac{w_jw_1}{k_jk_1} \right)^2  \right) k_j^2 \\
         & = \frac{1}{W^2 }  \sum_{j=1}^J \frac{w_j^2}{k_j}  \sum_{j=1}^J \frac{w_j^2}{k_j} = y^2
    \end{split}
    \nonumber
\end{equation}

Finally, since we have found $tr(A^2)$ and $tr(B^2)$, the last step in finding $t$ is to find $tr(2AB)$. 

\begin{equation}
    \begin{split}
        2AB  & = 2 \times \mathbf{\Phi}^{-1} \frac{1}{W}\begin{bmatrix}
            \frac{w_1}{k_1}\mathbf{1}_1 \\
            \vdots \\
            \frac{w_j}{k_j}\mathbf{1}_j
        \end{bmatrix} \begin{bmatrix}
            \frac{w_1}{k_1}\mathbf{1}_1' & \dots & \frac{w_j}{k_j}\mathbf{1}_j'  
        \end{bmatrix} \\
        & = \frac{2}{W} \begin{bmatrix}
            \mathbf{\Phi}_1^{-1} \mathbf{1}_1\mathbf{1}_1'\left(\frac{w_1}{k_1}\right)^2 & & \\
            & \ddots & \\
            & & \mathbf{\Phi}_j^{-1} \mathbf{1}_j\mathbf{1}_j'\left(\frac{w_j}{k_j}\right)^2
        \end{bmatrix} \\
        & = \frac{2}{W} \begin{bmatrix}
            \left(\frac{w_1}{k_1}\right)^3\mathbf{1}_1\mathbf{1}_1' & & \\
            & \ddots & \\
            & & \left(\frac{w_j}{k_j}\right)^3\mathbf{1}_j\mathbf{1}_j'
        \end{bmatrix}
    \end{split}
    \nonumber
\end{equation}

Now $tr(2AB)$ is:

\begin{equation}
    tr(2AB) = \frac{2}{W}tr\left(\left(\frac{w_j}{k_j}\right)^3\mathbf{1}_j\mathbf{1}_j'\right) = \frac{2}{W} \sum_{j=1}^J\times k_j \times \left(\frac{w_j}{k_j}\right)^3 = \frac{2}{W}\sum_{j=1}^J\frac{w_j^3}{k_j^2}
    \nonumber
\end{equation}

All together: $t = tr(QQ) = tr(A^2) + tr(B^2) - tr(2AB)$ which equals:

\begin{equation}
    t = \sum_{j=1}^J \frac{w_j^2}{k_j^2} + \sum_{j=1}^J \frac{k_j-1}{(\omega^2 + (1-\rho)\sigma_j^2)^2} + y^2 - \frac{2}{W}\sum_{j=1}^J\frac{w_j^3}{k_j^2}.
    \nonumber
\end{equation}


